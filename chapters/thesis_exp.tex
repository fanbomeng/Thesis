%%
%% This is file `template.tex',
%% generated with the docstrip utility.
%%
%% The original source files were:
%%
%% nddiss2e.dtx  (with options: `template')
%% 
%% This is a generated file.
%% 
%%  Copyright (C) 2004-2005 Sameer Vijay
%% 
%%  This file may be distributed and/or modified under the
%%  conditions of the LaTeX Project Public License, either
%%  version 1.2 of this license or (at your option) any later
%%  version. The latest version of this license is in
%%     http://www.latex-project.org/lppl.txt
%% 
%% 
%% ==============================================================
%% 
%% Notre Dame's Dissertation document class by Sameer Vijay
%% that adheres to the University of Notre Dame guidelines
%% published in Spring 2004.
%% 
%% Please send any improvements/suggestions to :
%%     Shari Hill, Graduate Reviewer.
%%     shill2@nd.edu
%% 
%% For documentation on how to use nddiss2e class, process the
%% file nddiss2e.dtx through LaTeX.
%% 
%% ==============================================================
%% 
%%\ProvidesFile{template.tex}
 %%   [2013/04/16 v3.2013^^J%
  %%   Template file for NDdiss2e class by Sameer Vijay and updated by Megan Patnott^^J]
\documentclass[final,numrefs,sort&compress,noinfo]{nddiss2e}
                     % One of the options draft, review, final must be chosen.
                     % One of the options textrefs or numrefs should be chosen
                     % to specify if you want numerical or ``author-date''
                     % style citations.
                     % Other available options are:
                     % 10pt/11pt/12pt (available with draft only)
                     % twoadvisors
                     % noinfo (should be used when you compile the final time
                     %         for formal submission)
                     % sort (sorts multiple citations in the order that they're
                     %       listed in the bibliography)
                     % compress (compresses numerical citations, e.g. [1,2,3]
                     %           becomes [1-3]; has no effect when used with
                     %           the textrefs option)
                     % sort&compress (sorts and compresses numerical citations;
                     %           is identical to sort when used with textrefs)

\usepackage{multirow}
\usepackage{subfigure}
\usepackage{hyperref}
\usepackage{slashed}
%\usepackage{cite}
\usepackage[numbers,sort&compress]{natbib}
\usepackage{subfigure}
\usepackage{amsmath}
\usepackage{amssymb}
\usepackage{hepnicenames}
\usepackage{longtable}
\usepackage{threeparttable}
\usepackage{footnote}
\usepackage{booktabs, caption, makecell}
\providecommand{\mt}{\ensuremath{M_\mathrm{T}}\xspace}
\newcommand{\Htt}{\ensuremath{\mathrm{H} \to \tau \tau}\xspace}
\newcommand{\ETmiss}{\ensuremath{\mathrm{E}_{T}^{miss}}\xspace}
\newcommand{\Htetmu}{\ensuremath{\mathrm{H} \to \tau_{e} \tau_{\mu}}\xspace}
\newcommand{\Hteth}{\ensuremath{\mathrm{H} \to \tau_{e} \tau_{\textrm{h}}}\xspace}
\newcommand{\Het}{\ensuremath{\mathrm{H} \to e \tau}\xspace}
\newcommand{\Hmt}{\ensuremath{\mathrm{H} \to \mu \tau}\xspace}
\newcommand{\Hmue}{\ensuremath{\mathrm{H} \to \mu \tau_{e}}\xspace}
\newcommand{\Hmuhad}{\ensuremath{\mathrm{H}\to \mu\tau_{h}}\xspace}
\newcommand{\Hehad}{\ensuremath{\mathrm{H} \to e \tau_{h}}\xspace}
\newcommand{\Hetaumu}{\ensuremath{\mathrm{H} \to e \tau_{\mu}}\xspace}
\newcommand{\Hemu}{\ensuremath{\mathrm{H} \to e \tau_{\mu}}\xspace}
\newcommand{\mue}{\ensuremath{\mu \tau_{e}}\xspace}
\newcommand{\emu}{\ensuremath{e \tau_{\mu}}\xspace}
\newcommand{\muhad}{\ensuremath{\mu \tau_{h}}\xspace}
\newcommand{\ehad}{\ensuremath{e \tau_{h}}\xspace}
\newcommand{\met}{\ensuremath{\cancel{\it{E}}_{T}}\xspace}
\newcommand{\mcol}{\ensuremath{M_{col}}\xspace}
\newcommand{\mvis}{\ensuremath{M_{vis}}\xspace}
\newcommand{\msig}{\ensuremath{100\:\GeV< M_{collinear} < \: 150\:\GeV}\xspace}
\newcommand{\tauh}{\ensuremath{\tau_{h}}\xspace}
\newcommand{\wjets}{\ensuremath{W+\textrm{jets}}\xspace}
\newcommand{\zjets}{\ensuremath{Z+\textrm{jets}}\xspace}
\newcommand{\pt}{\ensuremath{p_{T}}\xspace}
\newcommand{\ttbar}{\ensuremath{\Ptop\APtop}\xspace}
\newcommand{\NA}{\ensuremath{\--}\xspace}
\newcommand\tabitem{\makebox[1em][r]{\textbullet~}}



 
\usepackage[flushleft]{threeparttable}



\begin{document}

\frontmatter         % All the items before Chapter 1 go in ``frontmatter''

\title{SEARCH FOR LEPTON FLAVOUR VIOLATING DECAYS OF THE HIGGS BOSON}
\advisor{Colin Jessop}
\author{Fanbo Meng}
\work{Dissertation} % or \work{Thesis}
\degaward{Doctor of Philosophy}
\department{Physics}
\maketitle

 % You must use either the \makecopyright option or the \makepublicdomain option.
\copyrightholder{Fanbo Meng} % See template or documentation for
\copyrightyear{2019}           % other copyright options.
%\copyrightlicense{CC-BY-4.0}
\makecopyright

% uncomment out \makecopyright
% \makepublicdomain   % Uncomment this to make your work public domain

% Including an abstract is optional for a master's thesis, and required for a
% doctoral dissertation.

\begin{abstract}

A search for lepton flavour violation Higgs decay in the $H \to \mu\tau_{h}$ and $H\to e\tau_{h}$ in which tau leptons decay hadronically is presented. The search of tau lepton hadronic decay channels and the searches that are combined with tau lepton leptonic decays are presented. The $H \to e \tau_{h}$ search utilizes the 2012 proton-proton collision dataset at LHC with an integrated luminosity of 19.7 $fb^{-1}$ at a center-of-mass energy of 8 TeV collected by CMS experiment. No significant excess was observed. The upper limits on branching fraction is $B<0.69$ and the corresponding Yukawa coupling is set as $Y_{e\tau}<2.4\times10^{-3}$ at 95\% CL. For the $\Hmuhad$ search, the full 2016 proton-proton collision dataset with an integrated luminosity of 35.9 $fb^{-1}$ at a center-of-mass energy of 13 TeV was used. No significant excess was observed. The upper limits on branching fraction $H\to \mu\tau$ is $B<0.25\%$ and the corresponding Yukawa coupling is set as $Y_{\mu\tau}<1.43\times10^{-3}$ at 95\% CL. 




\end{abstract}

%                         % Either place the text between begin/end, or
% \include{abstract}  % put it in a file to be included

% Including a dedication is optional.
% \renewcommand{\dedicationname}{\mbox{}} % Replace \mbox{} if you want
                                           % something else. It must be in
                                           % all caps, and doing so is your
                                           % responsibility.
\begin{dedication}
To my family 
\end{dedication}

\tableofcontents
{%
\let\oldnumberline\numberline%
\renewcommand{\numberline}{\figurename~\oldnumberline}%
\listoffigures%
}

%\listoffigures
{%
\let\oldnumberline\numberline%
\renewcommand{\numberline}{\tablename~\oldnumberline}%
\listoftables%
}

\begin{acknowledge}

First I would like to thank my adviser Prof.Colin Jessop for the constant guidance, help and support in the past years in both research and everyday life. Thanks for Colin's advises step by step towards being a qualified researcher and great helps and understanding when I was in difficulties. I would also like to thank the help and accompanies from my friends at Notre Dame and CERN. Thanks for making my life colorful. Thanks for the high energy group providing me the opportunity of working at CERN. The CMS is a great collaboration with the best research environment, researchers and collaborators. 

Finally I would like to thank the love and supports from my family.  

 


\end{acknowledge}
 %                       % Use one of the two choices to add acknowledge text
 % \include{acknowledgement}

\mainmatter
 % Place the text body here.
 % \include{chapter-one}
 % Begin each chapter with \chapter{TITLE}. Chapter titles must be in all caps
 % and ensuring that they are is your responsibility.



\appendix

 % If you have appendices, add them here.
 % Begin each one with \chapter{TITLE} as before- the \appendix command takes
 % care of renaming chapter headings and creates a new page in the Table of
 % Contents for them.
 % \include{appendix-one}

\backmatter              % Place for bibliography and index
\bibliographystyle{unsrtnat}
%%\bibliographystyle{nddiss2e}
\bibliography{thesis}           % input the bib-database file name


\end{document}

%%
%%\endinput
%%
%% End of file `template.tex'.
