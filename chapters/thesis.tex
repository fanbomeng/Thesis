%%
%% This is file `template.tex',
%% generated with the docstrip utility.
%%
%% The original source files were:
%%
%% nddiss2e.dtx  (with options: `template')
%% 
%% This is a generated file.
%% 
%%  Copyright (C) 2004-2005 Sameer Vijay
%% 
%%  This file may be distributed and/or modified under the
%%  conditions of the LaTeX Project Public License, either
%%  version 1.2 of this license or (at your option) any later
%%  version. The latest version of this license is in
%%     http://www.latex-project.org/lppl.txt
%% 
%% 
%% ==============================================================
%% 
%% Notre Dame's Dissertation document class by Sameer Vijay
%% that adheres to the University of Notre Dame guidelines
%% published in Spring 2004.
%% 
%% Please send any improvements/suggestions to :
%%     Shari Hill, Graduate Reviewer.
%%     shill2@nd.edu
%% 
%% For documentation on how to use nddiss2e class, process the
%% file nddiss2e.dtx through LaTeX.
%% 
%% ==============================================================
%% 
%%\ProvidesFile{template.tex}
 %%   [2013/04/16 v3.2013^^J%
  %%   Template file for NDdiss2e class by Sameer Vijay and updated by Megan Patnott^^J]
\documentclass[final,numrefs,sort&compress,noinfo]{nddiss2e}
                     % One of the options draft, review, final must be chosen.
                     % One of the options textrefs or numrefs should be chosen
                     % to specify if you want numerical or ``author-date''
                     % style citations.
                     % Other available options are:
                     % 10pt/11pt/12pt (available with draft only)
                     % twoadvisors
                     % noinfo (should be used when you compile the final time
                     %         for formal submission)
                     % sort (sorts multiple citations in the order that they're
                     %       listed in the bibliography)
                     % compress (compresses numerical citations, e.g. [1,2,3]
                     %           becomes [1-3]; has no effect when used with
                     %           the textrefs option)
                     % sort&compress (sorts and compresses numerical citations;
                     %           is identical to sort when used with textrefs)

\newcommand{\tth}{\ensuremath{\mathrm{t}\overline{\mathrm{t}}\mathrm{H}}\xspace}
\newcommand{\ttw}{\ensuremath{\mathrm{t}\overline{\mathrm{t}}\mathrm{W}}\xspace}
\newcommand{\ttz}{\ensuremath{\mathrm{t}\overline{\mathrm{t}}\mathrm{Z}}\xspace}
\newcommand{\ttv}{\ensuremath{\mathrm{t}\overline{\mathrm{t}}\mathrm{V}}\xspace}
\newcommand{\ttbar}{\ensuremath{\mathrm{t}\overline{\mathrm{t}}}\xspace}
\newcommand{\pt}{\ensuremath{\mathrm{p}_{\mathrm{T}}}\xspace}
\newcommand{\met}{\ensuremath{\mathrm{E}^{\mathrm{miss}}_{\mathrm{T}}}\xspace}
\newcommand{\ht}{\ensuremath{\mathrm{H}^{\mathrm{miss}}_{\mathrm{T}}}\xspace}
\newcommand{\ptRatio}{\ensuremath{\pt^\text{ratio}}\xspace}
\newcommand{\miniIso}{\ensuremath{I_\text{mini}}\xspace}

\usepackage{multirow}
\usepackage{subfigure}
\usepackage{hyperref}
\begin{document}

\frontmatter         % All the items before Chapter 1 go in ``frontmatter''

\title{EVIDENCE FOR A STANDARD MODEL HIGGS BOSON PRODUCED IN ASSOCIATION WITH A TOP QUARK PAIR AND DECAYING TO LEPTONS}
\author{Charles Mueller}           % Author's name
\work{Dissertation}             % ``Dissertation'' or ``Thesis''
\degaward{Doctor of Philosophy}         % Degree you're aiming for. Should be one of the following options:
\advisor{Kevin Lannon}          % Advisor's name
\department{Physics}       % Name of the department

\maketitle           % The title page is created now

 % You must use either the \makecopyright option or the \makepublicdomain option.
\copyrightholder{Charles Mueller} % If you're not the copyright holder
\copyrightyear{2017}   % If the copyright is not for the current year
\makecopyright

% uncomment out \makecopyright
% \makepublicdomain   % Uncomment this to make your work public domain

% Including an abstract is optional for a master's thesis, and required for a
% doctoral dissertation.

\begin{abstract}
A search for the standard model Higgs boson produced in association with a top quark pair is presented, using the full pp collision
dataset corresponding to an integrated luminosity of 35.9 fb$^{-1}$ collected by the CMS experiment at a center of mass energy of $\sqrt{s}$ = 13 TeV.
MVA-based event reconstruction techniques are used to identify final states where the Higgs boson decays to either a W, Z or tau pair by
selecting events with two isolated same-sign leptons, and b-jets. The observed best-fit \tth signal strength is 1.7$^{+0.6}_{-0.5}$ times the
Standard Model prediction, corresponding to a significance of 3.3 standard deviations above the background-only hypothesis. The observed 95$\%$ CL
upper limit on the signal strength is 2.9 times the Standard Model prediction, compared to the expected upper limit of 1.0$^{+0.5}_{-0.3}$. 

\end{abstract}

%                         % Either place the text between begin/end, or
% \include{abstract}  % put it in a file to be included

% Including a dedication is optional.
% \renewcommand{\dedicationname}{\mbox{}} % Replace \mbox{} if you want
                                           % something else. It must be in
                                           % all caps, and doing so is your
                                           % responsibility.
\begin{dedication}
To my parents, Charles and Toni. 
\end{dedication}

\tableofcontents
\listoffigures
\listoftables

 % Including a list of symbols is optional.
 %% \renewcommand{\symbolsname}{newsymname} % Replace ``newsymname'' with
                                            % the name you want, and uncomment
                                            % The name must be in all caps,
                                            % and ensuring this is your
                                            % responsibility
 % \begin{symbols}
 % \end{symbols}
 %                       % Use one of the two choices to add symbols text
 % \include{symbols}

 % Including a preface is optional.
 %% \renewcommand{\prefacename}{ } % If you want another Preface name, add
                                   % something else, and uncomment.
                                   % The name must be in all caps, and
                                   % ensuring this is your responsibility.
 % \begin{preface}
 % \end{preface}
 %                       % Use one of the two choices to add preface text
 % \include{preface}

 % Including an acknowledgements section may or may not be optional. It's hard to
 % tell from the information available in Spring 2013.
 %% \renewcommand{\acknowledgename}{ } % If you want another Acknowledgement name
                                       % add something else, and uncomment
                                       % The name must be in all caps, and
                                       % ensuring this is your responsiblity.
\begin{acknowledge}
I would first like to acknowledge my advisor Kevin Lannon, whose support and guidance helped me through my graduate school experience.
Thanks to Kevin's advising, I was regularly engaged in useful, interesting and visible projects.
Kevin taught me the principles of scientific investigation, and I am confident that any further success I enjoy will be in part
due to the advising I received during my time at Notre Dame.

I would also like to acknowledge the faculty and staff in the High Energy Physics group at Notre Dame. In addition to my advisor,
Mike Hildreth, Colin Jessop, and other CMS faculty members helped create an effective and impactful research effort on CMS.
Thanks to their generous support, I was able to spend several years at CERN in an intellectually
stimulating environment that facilitated my transition into a researcher. 
I must also express my gratitude to the Notre Dame community,
especially the Physics Department staff: Sherry Herman, Shelly Goethals, and Susan Baxmeyer, who made Notre Dame and South Bend
feel like home. 

The measurements presented in this dissertation would not be possible without the efforts of thousands
of dedicated scientists, engineers and students working on the LHC and CMS experiment. 
I am grateful for the time I spent working in the CMS Trigger Studies Group,
where I learned both the nuances of creating and operating sophisticated software, and, more importantly, the nuances of working with other people.
A special thanks is due to Andrea Bocci, Tulika Bose, Aram Avestiyan, and Roberta Acridiacono.
More directly, this work is the result of the collaboration of many talented scientists working on \tth:
Wuming Luo, Christopher Neu, Matthias Wolf, Jason Slaunwhite, Marco Peruzzi, Francesco Romeo, Binghuan Li, and Giovanni Petrucciani. 
Special credit is due to Geoffrey Smith, the friend, officemate and postdoc whom I worked with most. 

During my time at CERN, I was fortunate enough to form friendships with special people who set examples
in kindness and scientific aptitude that I still do my best to follow. 
In addition to those mentioned above, this includes Justin Pilot, Christine McClean, Ted Kolberg,
Rachel Yohay, and Sean Flowers.
A special acknowledgement is due to my camrades de chambre and now close friends at Boulevard des Philosophes:
Andrea Tognina, Charlie Goodlake, Benjamin Tannenwald, and Johannes Fexer.
I will not soon forget our days on the lake, nights in Geneva, or adventures in the Alps.  
This circle of friends made work and play far more enjoyable
than I could have anticipated.
The same is true of my friends/classmates at Notre Dame:
Andrew Brinkerhoff, Joseph Hagmann, Nil Valls, Doug and Tessa Berry,
Anna Woodard, Nabarun Dev, Fanbo Meng, and Anthony Ruth. A special thanks is in order to my long-time officemate and
good friend Michael Planer, whom I enjoyed many long conversations, only some of which pertained to physics. 
I am especially grateful for Diane Polydoris, who was patient with me when I was stubborn, kind to me when I was rude,
and understanding when I was upset. I am still learning to appreciate the extent to which her support has grounded and helped me
over the years.

Finally, I thank my family, who supported my ambitions when success was uncertain.

\end{acknowledge}
 %                       % Use one of the two choices to add acknowledge text
 % \include{acknowledgement}

\mainmatter
 % Place the text body here.
 % \include{chapter-one}
 % Begin each chapter with \chapter{TITLE}. Chapter titles must be in all caps
 % and ensuring that they are is your responsibility.

 % If you have appendices, add them here.
 % Begin each one with \chapter{TITLE} as before- the \appendix command takes
 % care of renaming chapter headings and creates a new page in the Table of
 % Contents for them.
 % \include{appendix-one}

\backmatter              % Place for bibliography and index
\bibliographystyle{unsrtnat}
%%\bibliographystyle{nddiss2e}
\bibliography{thesis}           % input the bib-database file name


\end{document}

%%
%%\endinput
%%
%% End of file `template.tex'.
