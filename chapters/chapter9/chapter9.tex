
\chapter{Conclusion}

The operation of LHC opens a new era for experimental particle physics search with colliders. The great performance of the general purpose detector CMS provides the possibilities to fully explore new physics. Through out this dissertation, the theory background for the search of lepton flavour violation Higgs decay, an introduction of LHC and CMS, the detailed information about the data sets used, the reconstruction methods for the events in CMS and the search in two channels of LFV Higgs decays are presented. The focus is the physics search. 

  
The search for standard model Higgs lepton flavour violation decay on the LHC with CMS detecter. One search is on $\Hehad$ with an integrated luminosity of $19.7~fb^{-1}$ at center-of-mass energy 8 TeV dataset and another search is on $\Hmuhad$ with $35.9~fb^{-1}$ at 13 TeV. The results from both of the searches are presented, also the final results which combined with search for tau leptons decay electronically with respect to the tau hadronic decay ones. The $H \to e \tau$ search gave the upper limit from the direct measurement for the first time. The combined limit on the branching fraction in $H \to e\tau$ is $B<0.69$ at 95\% CL and on Yukawa coupling as $Y_{e\tau}<2.4\times10^{-3}$. The search on $\Hmuhad$ with $35.9~fb^{-1}$ 13 TeV is also presented. Even though the excess observed in 8 TeV search in $H \to \mu \tau$ channel is gone, the combined result from 13 TeV search is the tightest limit so far. In $\Hmuhad$, a combined limit with the $H \to \mu \tau_{e}$ channel on the branching ratio of $H \to \mu\tau$ is $B<0.25\%$ and the upper limit on Yukawa coupling is $Y_{\mu\tau}<1.43\times10^{-3}$.   

The latest search for 125 GeV Higgs LFV decay $H \to \mu \tau$ is presented in the dissertation. While for the $H \to e \tau$ search, the 13 TeV results published together with $H \to \mu \tau$ search here~\cite{Sirunyan2018}. Though no significant excesses are observed in both $\mu\tau$ and $e\tau$ channel, compared with the $H \to e \mu$ channel, there is still huge room to explore in this topic. Whether or not LFV is observed, this is an important topic which has big influences on new physics searches.  









