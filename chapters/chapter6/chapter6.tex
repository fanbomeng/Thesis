%
% Chapter 6
%

%% I need to describe muon ID in chaper 4.2 and Tau decay mode finding, tau MVA ID around 4.2 too 
\chapter{LFV event selection}
For both 8TeV analysis $H\rightarrow e\tau_h$ channel and 13TeV analysis  $H\rightarrow\mu\tau_h$, events are selected in several steps. The loose selection on the different IDs, energy, geometry parameters of the analysis related objects are applied. In both  $H\rightarrow e\tau_h$ and $H\rightarrow\mu\tau_h$ analysis, cut-based analyses are applied. In $H\rightarrow\mu\tau_h$, a multivariate analysis with Boosted decision tree (BDT) is exploited to provide more sensitive results. 



\section{\texorpdfstring{$H\rightarrow\mu\tau_h$}{Lg}}
\subsection{Loose selection}
In $H\rightarrow\mu\tau_h$ events, tau leptons from signal events decay hadronically. SM higgs is much heavier than the LFV decay products $\mu$ and $\tau$, so $\mu$ and $\tau$ are expected to have high $P_{T}$. Since the decay products are boosted, a cut on the $\Delta R>0.3$ is applied. $\Delta$R is defined as $\Delta R=\sqrt(\Delta phi^{2}+\Delta eta^{2})$. Higgs has no charge, so $\mu$ and $\tau$ candidates are required to have opposite sign of charges. Further, the events with additional $\mu$ and $\tau$ that pass a loose selection, the events with jets that are identified by the combined secondary vertex(CSVv2) b-tagging algorithm \cite{btag_ago} as a b quark jets will be vetoed. Muons from signal events are boosted and isolated, as mentioned in previous chapter, the trigger HLT\_IsoMu24 or HLT\_IsoTkMu24 is used. The trigger select isolated muons that have energy higher than 24GeV. An further $P_{T}$ cut on reconstructed $\mu$,  $P_{T}>26$GeV and $|\eta|<2.4$ are applied. Muons are required to pass the recommended Medium muon ID(chapter 4.2). For the LHC data 2016, running period BCDEF, ICHEP medium muon ID is applied(table~\ref{tbl:ICHEPMedID}), for data running period G and H, also the monte Carlo samples, standard medium muon ID(table~\ref{tbl:standardMedID})are applied to achieve the best performance for muon identification.

Hadronic taus are required to have $P_T>30$GeV,$|\eta|<2.3$, passing old tau decay mode finding(chapter 4.2), an MVA based tight tau isolation ID(Chapter 4.2) and tau discriminators against electrons and muons. These discriminators are very loose MVA based rejection against electrons and cut-based tight rejection against muons.  



Events in the analysis are divided into four categories based on the number of jets in an event. In 2-jets category, it is furthered divided into 2 categories,  2-jet gluon gluon fusion higgs production(ggH) category and 2-jet vector boson fusion(VBF) category based on the value of 2 jets invariant mass($M_{jj}$) . In 0-jet category, the signal mainly comes from ggH. In 1-jet category, the dominant signal production mode is also ggH, but with a boosted jet associated with the production, some of the VBF higgs signal also shows up in this category. In the 2-jet ggH category, signal evens mainly come from ggH and in 2-jet VBF, VBF production dominants the production mode.  The following is a more detailed list of the selection condition in each categories.

\begin{enumerate}
\item[{\bf 0-jet:}] No events have jets pass the loose PF ID and  with jet $P_T>30$ GeV, $|\eta|<4.7$.
\item[{\bf 1-jet:}] Events with one jet passes losse PF ID and jet $P_T>30$ GeV, $|\eta|<4.7$.
\item [{\bf 2-jets ggH:}] Events have two jets passing loose PF ID, $P_T>30$ GeV, $|\eta|<4.7$ and a requirement on the invariant mass of the two jets, $\textrm{M}_{jj}<550GeV$. 
\item [{\bf 2 jets VBF:}] Events with two jets pass loose PF ID. Jets $P_T>30$ GeV, $|\eta|<4.7$ and $\textrm{M}_{jj}>550 GeV$ are required. 
\end{enumerate} The threshold on $\textrm{M}_{jj}$ has been optimized to give the best expected exclusion limits.


\subsection{Cut-based analysis}
With the loose selection, including the categorization, a further cut-based selection strategy is applied. Variables that can help distinguish signal from background used in this analysis are $P_{T}^{\mu}$, $P_{T}^{\tau}$ and $M_{T}(\tau_{h})$. The lepton $P_{T}$ variables are very powerful background discriminant variables, but it will also cause the problem that signal picks under the background. Leptons from signal process  incline to have higher $P_{T}$ values, by cutting tighter on the lepton $P_{T}$, more background events can be removed. However this will also reshape some of the backgrounds, making them peak closer under the signal so that signal processes will be affected more by the background statistics fluctuation. In the $H\rightarrow e\tau_h$ analysis, the effect of cutting hard on lepton $P_{T}$ will be shown. So in $H\rightarrow\mu\tau_h$ search, lepton $P_{T}$ variables are kept at loose values and tune on other variables to achieve better signal significance. 

Tuning process is done only with Monte Carlo samples to not double use data. The final results are extract from data, so the double use of data may cause biases on the results. Misidentified lepton background estimated with full data driven method, is replaced with the semi data driven estimation in tuning. In $\Hmuhad$  channel, the variables tuned are $\textrm{M}_{jj}$ and $M_{T}(\tau_{h})$. Cuts have been optimized to have the most stringent expected limits with the Asimov dataset. If loose value of the cut gives same expected limits as the tight one, then chose the loose cut value to have more statistics. Examples of obtaining the limits are shown in Fig. \ref{fig:optMT}, $M_T(\tau_{h})$ for the different categories and Fig.~\ref{fig:optVBFmass}  for the optimization of $\textrm{M}_{jj}$ in 2-jets categories. With this method and criteria, the cut optimized for $\Hmuhad$ is shown in Table.~\ref{tab:Mhadcategories}


\begin{figure}[htbp] 
     \centering
     \subfigure[0 jet]{ \includegraphics[width=0.4\textwidth]{chapter6/Tuning/ggtMtToPfMet_type1.pdf}}
     \subfigure[1 jet]{ \includegraphics[width=0.4\textwidth]{chapter6/Tuning/boosttMtToPfMet_type1.pdf}}\\
     \subfigure[2 jets, gg-enriched]{ \includegraphics[width=0.4\textwidth]{chapter6/Tuning/vbf_ggtMtToPfMet_type1.pdf}}
     \subfigure[2 jets, VBF-enriched]{ \includegraphics[width=0.4\textwidth]{chapter6/Tuning/vbf_vbftMtToPfMet_type1.pdf}}
     \caption{Expected limits based on an Asimov dataset as a function of $M_T(\tau, MET)$ for the different categories.}
     \label{fig:optMT}
\end{figure}

\begin{figure}[!tbp] 
\centering
\includegraphics[width=0.4\textwidth]{chapter6/Tuning/vbf_vbfvbfMass.pdf}
\caption{Expected limits based on an Asimov dataset as a function of $M_{jj}$ for the 2 jet categories.}
\label{fig:optVBFmass}
\end{figure}



\begin{table}[hbtp]
  \begin{center}
  \caption{Selection criteria for each event category after cut
    optimization, for the $\Hmuhad$ channel}
  \begin{tabular}{l} \hline
  {\bf 0-jet category} \\ \hline
  \tabitem $\pt^{\mu}>26GeV$, $\pt^{\tau}>30GeV$\\
  \tabitem $M_T(\tau)<105GeV$ \\
  \tabitem No jets with $\pt^{jet}>30 GeV$, $|\eta|<4.7$, LooseID \\ \hline
 {\bf 1-jet category} \\ \hline
  \tabitem $\pt^{\mu}>26GeV$, $\pt^{\tau}>30GeV$ \\
  \tabitem $M_T(\tau)<105GeV$ \\
  \tabitem One jet  with $\pt^{jet}>30 GeV$, $|\eta|<4.7$, LooseID
  \\ \hline
  {\bf 2-jet, gg-enriched category} \\ \hline
  \tabitem $\pt^{\mu}>26GeV$, $\pt^{\tau}>30GeV$ \\
  \tabitem $M_T(\tau)<105GeV$ \\
      \tabitem $\pt^{jet1}>30 GeV$,$\pt^{jet2}>30 GeV$
      $|\eta_{jet1}|<4.7$,$|\eta_{jet2}|<4.7$, LooseID\\
      \tabitem $M_{jj}<550GeV$\\
      \tabitem Two jets with $\pt^{jet}>30 GeV$, $|\eta|<4.7$, LooseID\\ \hline
  {\bf 2-jet, vbf-enriched category} \\ \hline
  \tabitem $\pt^{\mu}>26GeV$, $\pt^{\tau}>30GeV$ \\
  \tabitem $M_T(\tau)<85GeV$ \\
      \tabitem $\pt^{jet1}>30 GeV$,$\pt^{jet2}>30 GeV$
      $|\eta_{jet1}|<4.7$,$|\eta_{jet2}|<4.7$, LooseID\\
      \tabitem $M_{jj}>550GeV$\\
      \tabitem Two jets with $\pt^{jet}>30 GeV$, $|\eta|<4.7$, LooseID\\ \hline
  \label{tab:Mhadcategories}
\end{tabular}
\end{center}
\end{table}




\subsection{Multivariate analysis}
A Boost decision trees(BDT) method is chosen as the multivariate analysis method used in  $H\rightarrow\mu\tau_h$ search. It provides better sensitivity compared with the cut-based analysis. In this analysis, the BDT is provided by the TMVA package \cite{TMVAnote}. BDT method takes in signal and background datasets with a selected set of input variables. Input variables are the ones that show distinguishing power between signal and background.  The training output is a weight file, which contains a list of weights to indicate in percentage how likely an event is signal like with a give set of input variable values from that event. A more detail description of the BDT method is available in section \ref{BDTchaper}.  In this analysis, signal and background events are required to pass the loose selection criteria. All of the categories are combined. The signal events from gluon gluon fusion and vector boson fusion higgs production mode are mix by the weight with respect to their production cross section. The background sample used in the training is misidentified lepton background from the like sign region(Region II as in table **). The list of BDT input variables  is following and the distribution of the variables are shown in Fig.~\ref{fig:BDT_input_var_mutauhad}.

\begin{itemize}
\item Transverse mass between the $\tauh$ and $\ETmiss$, $M_{T}(\tau_{h})$.
\item Missing transverse energy, $\ETmiss$.
\item Pseudorapidity difference between the $\mu$ and the $\tauh$ candidate, $\Delta\eta(\mu, \tauh)$.
\item Azimutal angle between the $\mu$ and the $\tauh$, $\Delta\phi (\mu,\tauh)$.
\item Azimutal angle between the $\tauh$ and the $\ETmiss$, $\Delta\phi(\tau_h,\ETmiss)$.
\item Collinear mass, $\mcol$.
\item Muon $\pt$.
\item $\tau_h$ $\pt$.
\end{itemize}

\begin{figure}[htpb]\centering
 \includegraphics[width=0.315\textwidth]{chapter6/BDTvariable/LFV_preselection_collMass_type1_Fakes_PoissonErrors.pdf}
 \includegraphics[width=0.315\textwidth]{chapter6/BDTvariable/LFV_preselection_mPt_Fakes_PoissonErrors.pdf} \\
 \includegraphics[width=0.315\textwidth]{chapter6/BDTvariable/LFV_preselection_tPt_Fakes_PoissonErrors.pdf}
 \includegraphics[width=0.315\textwidth]{chapter6/BDTvariable/LFV_preselection_tMtToPfMet_type1_Fakes_PoissonErrors.pdf}  \\
 \includegraphics[width=0.315\textwidth]{chapter6/BDTvariable/LFV_preselection_tDPhiToPfMet_type1_Fakes_PoissonErrors.pdf}
 \includegraphics[width=0.315\textwidth]{chapter6/BDTvariable/LFV_preselection_type1_pfMetEt_Fakes_PoissonErrors.pdf} \\
 \includegraphics[width=0.315\textwidth]{chapter6/BDTvariable/LFV_preselection_m_t_DPhi_Fakes_PoissonErrors.pdf}
 \includegraphics[width=0.315\textwidth]{chapter6/BDTvariable/LFV_preselection_m_t_DEta_Fakes_PoissonErrors.pdf}
\caption{Distributions of the  input variables to the BDT for the \Hmuhad channel.}
 \label{fig:BDT_input_var_mutauhad}
\end{figure}


\begin{figure}[htbp] 
     \centering
     \subfigure[Signal sample variables correlation]{ \includegraphics[width=0.4\textwidth]{chapter6/CorrelationMatrixS.pdf}}
     \subfigure[Background sample variable correlation]{ \includegraphics[width=0.4\textwidth]{chapter6/CorrelationMatrixB.pdf}}\\
     \caption{Expected limits based on an Asimov dataset as a function of $M_T(\tau, MET)$ for the different categories.}
     \label{fig:BDTvarcorrelation}
\end{figure}


Fig.~\ref{fig:BDTvarcorrelation} shows BDT input variables in the signal and background training samples. The chosen input variables show low correlation in both samples. Fig.~\ref{fig:BDTovertraining} shows the TMVA overtraining checks. Training samples are divided into two groups, one for training and one for testing. After the training, the algorithm applies the training output to the testing sample to check if they are in good matching.

\begin{figure}[htbp] 
\centering
\includegraphics[width=0.6\textwidth]{chapter6/overtrain_BDT.pdf}
\caption{Overtraining checking for the BDT training in the TMVA package.}
\label{fig:BDTovertraining}
\end{figure}

\section{\texorpdfstring{$H\rightarrow e \tau_h$}{Lg}}

\subsection{Loose selection}
In $ \Hehad$ channel, the trigger used is $HLT\_Ele27\_WP80$, which applies an electron $\pt$ cut at 27GeV at HLT level. A further cut on electron $P_{T}>30GeV$ is applied. Electrons are required to have $|\eta_{e}<2.3|$ and $D_{z}<0.2 cm$. $D_{z}$ is the longitudinal impact parameter that shows the displacement between primary vertex and track path. Electrons are also requirement to pass the MVA based tight ID and cut based PF tight isolation  $PFiso<0.1$. Tau candidates are required to have $P_{T}>30GeV$, pseudorapidity in the range $|\eta^{\tau}<2.3|$ and the longitudinal impact parameter $D_{z}<0.2 cm$. Tau isolation used is the cut based tight tau isolation. In addition, tau candidates pass the tau Decay mode finding and tau discriminator against electrons and muons. The analysis also requires no extra isolated electrons with $\pt>10GeV$ and extra taus with $\pt>20GeV$. Tau and electron candidates are required to have opposite sign of charges and separate with an $\Delta R>0.4$ from any jets in the events with $\pt>30GeV$. All of the requirements contribute to the selections of good qualities candidates. The datasets are binned into three categories according to the number of jets in the events:
\begin{enumerate}
\item[{\bf 0-jet:}] No events have jets pass the loose PF ID and  with jet $P_T>30GeV$, $|\eta|<4.7$. This category enhances the gluon-gluon fusion contribution.
\item[{\bf 1-jet:}] Events with one jet passes loose PF ID and jet $P_T>30GeV$ , $|\eta|<4.7$. This category enhances the gluon-gluon fusion production with initial state radiation.
\item [{\bf 2 jets:}] Events with two jets pass loose PF ID and with jet $P_T>30$ GeV and $|\eta|<4.7$, This category contains both Higgs production mode and with an enhancement in VBF production mode. \end{enumerate}

With the preselection and binning of the jets numbers, the $\mcol$ distribution of $\Hehad$ in different categories are shown in Fig.~\ref{fig:etauCol_preselection}
\begin{figure}[hbtp]\centering
 \includegraphics[width=0.4565\textwidth]{chapter6/etauPlots/etau_preselection0jet.pdf}
 \includegraphics[width=0.4565\textwidth]{chapter6/etauPlots/etau_preselection1jet.pdf}
 \includegraphics[width=0.4565\textwidth]{chapter6/etauPlots/etau_preselection2jet.pdf}
 \caption{With loose selection conditions, the comparison of the observed collinear mass distributions with background from prediction. The shaded grey bands indicate the total background uncertainty.
The open histograms correspond to the expected signal distributions for $\mathcal{B}(\Hehad)=100\%$ in the  0-jet, 1-jet and 2-jet categories, respectively.}
\label{fig:etauCol_preselection}\end{figure}


\subsection{Cut-based analysis}
A set of kinematics variables are defined and set to further select signal events. In $\Hehad$ channel, similar to the $\Hmuhad$, muon and tau leptons from the signal events are highly boosted, so as $\pt$ variables plays an important role in distinguishing from background events and the separation in $\phi$ direction is bigger between $\mu$ and $\tau$ for signal events. Missing energy from the $\tau$ neutrino in hadronic decay is one of the characters in signal events, so $M_{T}(\tau_{h})$ is an useful variable. In two jets category, $M_{jj}$ also used as a cut variable.  The cuts have been optimized to have the most stringent expected limits with the Asimov dataset. The detailed cuts used for $\Hehad$ is shown in table.~\ref{tab:ehadcategories}.


\begin{table}[hbtp]
  \begin{center}
  \caption{Selection criteria for each event category after cut
    optimization, for the $\Hehad$ channel}
  \begin{tabular}{l} \hline
  {\bf 0-jet category} \\ \hline
  \tabitem $\pt^{e}>45GeV$, $\pt^{\tau}>30GeV$, $\Delta \phi_{e \tau}>2.3$ \\
  \tabitem $M_T(\tau)<70GeV$ \\
  \tabitem No jets with $\pt^{jet}>30 GeV$, $|\eta|<4.7$, LooseID \\ \hline
 {\bf 1-jet category} \\ \hline
  \tabitem $\pt^{e}>35GeV$, $\pt^{\tau}>40GeV$ \\
  \tabitem One jet  with $\pt^{jet}>30 GeV$, $|\eta|<4.7$, LooseID
  \\ \hline
  {\bf 2-jet category} \\ \hline
  \tabitem $\pt^{e}>35GeV$, $\pt^{\tau}>30GeV$ \\
  \tabitem $M_T(\tau)<50GeV$ \\
      \tabitem $\pt^{jet1}>30 GeV$,$\pt^{jet2}>30 GeV$
      $|\eta_{jet1}|<4.7$,$|\eta_{jet2}|<4.7$, LooseID\\
      \tabitem $\Delta\eta(jet1,jet2)>2.3$\\
      \tabitem $M_{jj}>400GeV$\\
      \tabitem Two jets with $\pt^{jet}>30 GeV$, $|\eta|<4.7$, LooseID \\ \hline
  \label{tab:ehadcategories}
\end{tabular}
\end{center}
\end{table}











