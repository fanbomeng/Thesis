
\chapter{INTROCUATION}

The goal of elementary particle physics is to understand the fundamental particles and the interactions among them. Based in the microscopic world, particle physics attempts to construct theories that try to understand phenomena, make predictions and finally come to a consistent and unified theory. 

The Standard Model(SM) provides the modern understanding of particle physics. It incorporates the knowledge from decades of research. Except for gravity, the SM describes interactions among elementary particles that includes three out of four fundamental forces. The last missing piece, predicted by the SM, the Higgs boson was discovered in the Large Hadron Collider by both of the general purpose detectors, the CMS and ATLAS detector in 2012. The Higgs boson is a unique and crucial component of the SM. The Higgs boson is incorporated in the SM through the Higgs mechanism, which provides masses to fermions while keeping the theory renormalizable under local gauge symmetry. If viewing the Higgs mechanism within a bigger picture, the spontaneous symmetry breaking of SU(2) group, it is also related to providing the masses to other gauge bosons that mediate the interactions. The discovery of the Higgs boson is a great achievement both theoretically and experimentally. Further studies about the properties of the newly discovered boson show good agreement with the predictions from the SM. Besides the great success of the SM, there are still a number of questions that beyond the scope of the SM, for example, failing to include gravity and describing dark matter that shows its signature in the astronomy observations. New physics models are needed to understand these questions beyond the SM.  

The newly discovered Higgs boson can be a great probe to the physics beyond the SM, which in turn can test and inspire new physics models. In the SM, each flavour of leptons in the three generations is assigned a unique lepton number that is conserved within each family. The violation of this conservation which is referred as lepton flavour violation(LFV) is forbidden in the SM but allowed in many new physics scenarios. Thus, any observation of LFV will be a clear signature of new physics. The discovery of the Higgs boson can open a new window towards this topic. Before LHC era, direct searches for LFV Higgs decay, for example $H \to \mu\tau$ instead of the SM process $H \to \tau\tau$ or $H \to \mu\mu$, was impossible. The upper limits on $B(H \to \mu \tau)$ and $B(H \to e \tau)$ of $O(10\%)$~\cite{Blankenburg:2012ex,Harnik2012pb} from the indirect searches like $\tau \to \mu$ and $\tau \to e$~\cite{Celis:2013xja}. In these indirect searches, the LFV Higgs decays can only occur in the processes involving loop diagrams and the LFV Higgs decays to $\mu\tau$ and $e\tau$ from indirect searches leave plenty of room to further probe this topic. While for $H \to e\mu$, there is already a strong constrain from $\mu \to e\gamma$ at $B(H\to e \mu)<O(10^{-9})$~\cite{TheMEG2016wtm}. The CMS experiment published the first direct search on $H \to \mu \tau$ with integrated luminosity of $19.7~ fb^{-1}$ at center-of-mass energy of 8 TeV. A 2.4 $\sigma$ excess was observed and the best fit branching ratio was found to be $B(\Hmuhad)=(0.84^{+0.39}_{-0.37})\%$ at 95\% CL~\cite{2015337}. A following up search from ATLAS experiment showed the upper limits in $H \to \mu \tau$ channel as $B(H \to \mu \tau)=(1.24^{+0.50}_{-0.35})\%$ at 95\% CL~\cite{Aad2015gha}.
  

This dissertation focuses on the first direct search in $\Hehad$ with integrated luminosity of $19.7 ~fb^{-1}$ at 8 TeV and the search in $\Hmuhad$ with integrated luminosity of $35.9~ fb^{-1}$ at 13 TeV with the CMS detector. The dissertation is organized as following. A brief summary of the SM, specifically the electro-weak sector, the Higgs mechanism and LFV in beyond the SM theories are presented in Chapter 2. Following the theory chapter is the one that describes LHC and CMS experiment in general and sub-detectors in CMS. Chapter 4 includes the information about datasets used in the analyses, which incorporate the data collected from the experiment and Monte Carlo(MC) simulations, also the re-construction procedure of the basic objects used in the analysis, like muon, tau and jet. The selections used in $\Hmuhad$ and $\Hehad$ analysis are described in Chapter 5. In Chapter 6, a detailed description of the background processes considered in analyses is presented.  In chapter 7, the event selection and signal  extraction is described. In Chapter 8, the results from $\Hmuhad$ and $\Hehad$ analysis are summarized and presented. In chapter 9, the results are interpreted in terms of LFV Yukawa couplings.





