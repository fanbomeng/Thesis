
\chapter{Introduction}

The goal of elementary particle physics is to understand the fundamental particles and the interactions among them in the universe. Based in the micro-scope, particle physics is building up theories that tries to understand the phenomenons and makes predictions and finally comes to a consistent and unified theory. 

Standard Model(SM) provides the modern understanding of particle physics. It incorporates the knowledges from decays of research. Besides gravity, SM describes three out of four fundamental forces and interactions among a table of elementary particles. The last missing piece, predicted by SM, the Higgs boson was discovered in Large Hadron Collider with both of the general purpose detectors, the CMS and Atlas detector in 2012. Higgs boson is a unique and crucial component of SM. Higgs boson is incorporated in SM through the Higgs mechanism, which provides masses to fermions and keeps the theory renormalizable under local gauge symmetry. If viewing Higgs mechanism within a bigger picture, the spontaneous symmetry breaking of SU(2) group, which also provides masses to other gauge bosons that mediate the interactions. The discovery of Higgs boson is a great achievement both theoretically and experimentally. Further studies about the properties of the new discovered boson shows good agreement with the predictions from SM. Besides the great success of SM, there are still a number of questions that beyond the scope of SM for example, failing to include gravity and describe Dark matter that shows its signature in the observations. New physics models are need to understand these questions beyond SM.  

The new discovered Higgs boson can be a great probe to the physics beyond SM, which can in turn test and inspire new physics models. Lepton flavour violation(LFV) is forbidden in SM but allowed in many new physics scenarios. Thus, any solid results of LFV will be a clear signature of new physics. The discovery of Higgs boson can open a new window towards this topic. Before LHC era, direct search for LFV Higgs decay is impossible. The upper limits on $B(\Hmuhad)$ and $B(\Hehad)$ of $O(10\%)$~\cite{Blankenburg:2012ex,Harnik2012pb} from the indirect searches like $\tau \to \mu$ and $\tau \to e$~\cite{Celis:2013xja}, leave big room to further probe this search directly with Higgs boson. While for $H \to e\mu$, there is already strong constrain from $\mu \to e\gamma$ at $B(H\to e \mu)<O(10^{-9})$~\cite{TheMEG2016wtm}. The CMS experiment published the first direct search on $\Hmuhad$ with integrated luminosity of $19.7~ fb^{-1}$ at center-of-mass energy of 8 TeV. A 2.4 $\sigma$ excess is observed and best fit branching ratio was found to be $B(\Hmuhad)=(0.84^{+0.39}_{-0.37})\%$ at 95\% CL~\cite{2015337}. A following up search from Atlas experiment show the upper limits in $\Hmuhad$ channel as $B(\Hmuhad)=(1.24^{+0.50}_{-0.35})\%$ at 95\% CL~\cite{Aad2015gha}.
  

This thesis focus on the first direct search in $\Hehad$ with integrated luminosity of $19.7 ~fb^{-1}$ at 8 TeV and a following up search in $\Hmuhad$ with integrated luminosity of $35.9~ fb^{-1}$ at 13 TeV with CMS detetor. The thesis is organized as following. A brief recaption of SM, specifically the electro-weak section, Higgs mechanism and LFV in beyond SM theories are presented in Chapter 2. Following the theory chapter is the one that describes LHC and CMS experiment, which also includes the descriptions of sub-detectors in CMS. In Chapter 4, the information about datasets used in the analyses. This includes the data collected from the experiment and Monte Carlo(MC) simulations, also the re-constructions procedures of the basic objets used in the analysis, like muon, tau and jets. The selections used in $\Hmuhad$ and $\Hehad$ analysis are shown in Chapter 5. In Chapter 6, a detail description of the background processes considered in analyses is presented. Boosted decision tree(BDT) method, statistics methods and nuisance parameters considered in the analyses are discussed in Chapter 7. In Chapter 8, the results from $\Hmuhad$ and $\Hehad$ analysis are summarized and presented. A conclusion of the research and thesis is in Chapter 9.





