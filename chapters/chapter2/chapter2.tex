
\chapter{Theory}

\section{Standard model}
The discovery of the Higgs boson in 2012 was the last missing piece in SM.  The particles in the SM are shown in Figure.~\ref{fig:SM_particles}. Besides the Higgs boson, there are three generations of leptons and quarks and four gauge bosons. 
Leptons and quarks are fermions that compose the observable matter currently known in the universe, while gauge bosons serve as the mediators of the interactions.  

The SM is a gauge theory, which describes three natural forces, the strong, electromagnetic and weak force by symmetry group $SU(3)_{C}\times SU(2)_{L}\times U(1)_{Y}$. The electroweak sector is described by the  $SU(2)_{L}\times U(1)_{Y}$. The L in the subscript of $SU(2)_{L}$ stands for the left chiral signature of the weak interaction and the Y in $U(1)_{Y}$ stands for the weak hypercharge. The strong interaction is described by $SU(3)_{C}$, in which the C stands for the colors of quarks.  Photons, gluons and $W^{\pm}$, Z bosons are the mediators of electromagnetic, strong and weak force respectively. The force of gravity is not included in the SM. Each particle in Figure.~\ref{fig:SM_particles} has its anti-particle or the anti-particle is the particle itself. The characters of the forces and mediators are summarized in Table.~\ref{Mediator_infor}

\begin{figure}[htbp] 
\centering
\includegraphics[width=0.8\textwidth]{chapter2/SM_particle_table.pdf}
\caption{Standard model particle group\cite{SM_particletable}}
\label{fig:SM_particles}
\end{figure}


\begin{table}[htp]
\caption{Mediators in standard model~\cite{Griffiths:111880}}
\begin{center}
\begin{tabular}{|c|c|c|c|c|}
\hline
Force & Mediator & Charge & Mass(GeV) & Range(m)   \\\hline
Strong                & g(8 gluons)                  & 0 & 0 & $10^{-15}$   \\\hline
Electromagnetic & $\gamma$(photon)     & 0 & 0 &   $\infty$   \\\hline
Weak                  &  $W^{\pm}$                &$\pm$1 & 80.379 &$10^{-18}$\\
                           & Z                                &  0         & 91.1876 &  $10^{-18}$ \\\hline
\end{tabular}
\end{center}
\label{Mediator_infor}
\end{table}%


The fermion fields in SM can be further categorized. In weak interaction, left-handed fermions are isodoublets, while right-handed fermions are isosinglets. Only left-handed fermions or right-handed anti-fermions participate in the weak interaction. Weak hypercharge is a combined quantity, which is defined by the electric charge Q and the third component of the weak isospin $I^{3}_{f}$, $Y=Q-I^{3}_{f}$. The weak interaction, together with Electromagnetic interaction is best described and understood by the electro-weak theory, which will be further discussed in the next section.  Quarks are described by SU(3), which are color triplets, while leptons are color singlets.   

The mediators in the interaction are represented by the gauge boson fields. $B_{\mu}$ is associated with symmetry $U(1)_{Y}$ and the corresponding generator is Y. $W_{\mu}^{1,2,3}$ are associated to $SU(2)_{L}$ symmetry with the generators $T^{a}$(a=1,2,3). The $T^{a}$ are the $2\times2$ Pauli matrices, while the $G_{\mu}^{1...,8}$ are associated with the $SU(3)_{c}$ symmetry and the corresponding generators are the Gell-Mann matrices.  The field strengths are expressed as 
\begin{equation}
  \begin{aligned}
G^{a}_{\mu v}&=\partial_{\mu}G^{a}_{v}-\partial_{v}G^{a}_{\mu}+g_{s}f^{abc}G^{b}_{\mu}G^{c}_{v}\\
W^{a}_{\mu v}&=\partial_{\mu}W^{a}_{v}-\partial_{v}W^{a}_{\mu}+g_{2}\epsilon^{abc}W^{b}_{\mu}W^{c}_{v}\\
B_{\mu v}       &=\partial_{\mu}B_{v}-\partial_{v}B_{\mu}
  \end{aligned}
\end{equation}
The $g_{s}$ and $g_{2}$ are the coupling constant of $SU(3)_{C}$ and $SU(2)_{L}$ respectively. With the requirement of local gauge invariable, covariant derivatives are widely used. The covariant derivative in the Lagrangian is a substitution of normal derivative. It specifies the interaction between matter and gauge fields while keeps the Lagrangian invariant under local transformation. An example of a covariant derivative acting on left-handed quarks in Lagrangian is expressed as
\begin{equation}
  \begin{aligned}
D_{\mu}\psi = (\partial_{\mu}-ig_{s}T_{a}G^{a}_{\mu}-ig_{2}T_{a}W^{a}_{\mu}-ig_{1}\frac{Y_{q}}{2}B_{\mu})\psi
  \end{aligned}
\end{equation} 

The masses of gauge bosons are generated through spontaneous symmetry breaking by applying the Higgs mechanism, which also enables one to introduce the fermion masses. If boson mass terms are directly added in the SM Lagrangian, the local symmetry $SU(2)\times U(1)$ will be destroyed~\cite{DJOUADI20081}. Local symmetry here refers to the symmetry that smoothly depends on space and time. In SM, the interaction between fermion field and scalars field is through Yukawa interaction. The Lagrangian of SM in $SU(3)_{C}\times SU(2)_{L}\times U(1)_{Y}$ symmetry without the mass terms and Yukawa interactions is given by
 \begin{equation}
  \begin{aligned}
L_{SM}=&-\frac{1}{4}G^{a}_{\mu v}G^{\mu v}_{a}-\frac{1}{4}W^{a}_{\mu v}W^{\mu v}_{a}-\frac{1}{4}B_{\mu v}B^{\mu v}+\bar{L}_{i}iD_{\mu}\gamma^{\mu}L_{i}\\
              &+\bar{e}_{R_{i}}iD_{\mu}\gamma^{\mu}e_{R_{i}}+\bar{Q}_{i}iD_{\mu}\gamma^{\mu}Q_{i}+\bar{\mu}_{R_{i}}iD_{\mu}\gamma^{\mu}\mu_{R_{i}}+\bar{R_{i}}iD_{\mu}\gamma^{\mu}d_{R_{i}}
  \end{aligned}
\end{equation} 






\subsection{Spontaneous symmetry breaking and Higgs mechanism}

In the SM, the electroweak sector follows the symmetry $SU(2)_{L}\times U(1)_{Y}$, which is spontaneously broken in the $SU(2)_{L}$ part. This mechanism plays an important role in giving masses to the mediator bosons in weak interaction and introducing the Higgs boson(Higgs).  The interactions of the scalar doublet with fermion fields through the Yukawa interactions are responsible for generating the masses of the fermions in SM. A good example of showing the concept of spontaneous symmetry breaking is with the $\phi^{4}$ theory which can be found in~\cite{Peskin:1995ev}.

In the SM, spontaneous symmetry breaking is introduced through the scalar doublet:
\[
\phi=
\begin{pmatrix}
\phi^{+}\\
\phi^{0}
\end{pmatrix}
=\frac{1}{\sqrt{2}}
\begin{pmatrix}
\phi_{2}-i\phi_{1}\\
\phi_{4}-i\phi_{3}
\end{pmatrix}
\]
The corresponding terms of the scalar doublet in the SM Lagrangian are expressed as
\begin{equation}
L_{s}=(D^{\mu}\phi)^{\dagger}(D_{\mu}\phi)-\mu^{2}\phi^{\dagger}\phi-\lambda(\phi^{\dagger}\phi)^{2}
\end{equation}
If $\mu^{2}<0$ and $\lambda>0$, the doublet field $\phi$ will have a none zero minimum value, which is called the vacuum expectation value(VEV). In SM, the VEV is shown in the following equation and measured to be 246 GeV.
\begin{equation}\label{vev}
\langle 0 | \phi | 0 \rangle   =
\begin{pmatrix}
0\\
\frac{v}{\sqrt{2}}
\end{pmatrix}
~~~\textrm{with} ~~~  v=
\bigg(-\frac{\mu^{2}}{\lambda}\bigg)^{1/2}
\end{equation}

When the SU(2) symmetry is spontaneously broken, the scalar doublet field $\phi$ can expand around the VEV at first order together with the Higgs field:
\begin{equation}\label{Higgs_vev_expansion}
\phi=
\begin{pmatrix}
\theta_{2}+i\theta_{1} \\
\frac{1}{\sqrt{2}}(v+H)-i\theta_{3}
\end{pmatrix}
=e^{i\theta_{a}(x)T^{a}(x)/v}
\begin{pmatrix}
0\\
\frac{1}{\sqrt{2}}(v+H)
\end{pmatrix}
\end{equation}
Taking the unitary gauge, the scalar field transforms as $\phi \to e^{-i\theta_{a}(x)T^{a}(x)/v}\phi$ and the term $(D^{\mu}\phi)^{\dagger}(D_{\mu}\phi)$ can be expended as:
\begin{equation}\label{Lag_scaler}
\begin{aligned}
(D^{\mu}\phi)^{\dagger}(D_{\mu}\phi)=&|(\partial_{\mu}-ig_{2}\frac{T_{a}}{2}W^{a}_{\mu}-\frac{i}{2}g_{1}B_{\mu})\phi|^{2}\\
                                                          =&\frac{1}{2}(\partial_{\mu}H)^{2}+\frac{1}{8}g^{2}_{2}(v+H)^{2}|W^{1}_{\mu}+iW^{2}_{\mu}|^{2}\\
                                                            &+\frac{1}{8}(v+H)^{2}|g_{2}W^{3}_{\mu}-g_{1}B_{\mu}|^{2}
\end{aligned}
\end{equation}
$W^{\pm}$ and Z can be expressed as following by re-grouping the terms in the above equation: 
\begin{equation}
W^{\pm}=\frac{1}{\sqrt{2}}(W_{1}\mp iW_{2}), Z_{\mu}=\frac{g_{2}W^{3}_{\mu}-g_{1}B_{\mu}}{\sqrt{g_{2}^{2}+g^{2}_{1}}},A_{\mu}=\frac{g_{2}W^{3}_{\mu}+g_{1}B_{\mu}}{\sqrt{g_{2}^{2}+g^{2}_{1}}},
\end{equation}
From the terms in Equation~\ref{Lag_scaler} that involving the mass of $W^{\pm}$ and Z boson, these vector bosons acquire the mass as
\begin{equation}
M_{W}=\frac{1}{2}vg_{2}~~~\textrm{and}~~~M_{Z}=\frac{1}{2}v\sqrt{g^{2}_{2}+g^{2}_{1}}
\end{equation}
while photon $A_{\mu}$ remains massless. The mixing of electromagnetic and weak interaction is often expressed in terms of Weinberg angle or weak mixing angle. The angle $\theta_{W}$ is defined as following:
\[
\begin{pmatrix}
\gamma    \\
Z^{0}
\end{pmatrix}
=
\begin{pmatrix}
cos\theta_{W}  & sin\theta_{W} \\
-sin\theta_{W}  & cos\theta_{W}
\end{pmatrix}
\begin{pmatrix}
B\\
W_{3}
\end{pmatrix}
\]
\begin{equation}
M_{Z}=\frac{M_{W}}{cos\theta_{W}}~~~ \textrm{and}~~~
cos\theta_{W}=\frac{g_{2}}{\sqrt{g_{2}^{2}+g_{1}^{2}}}
\end{equation}

The mass of the fermion can also be generated with the interaction between the scalar doublet $\phi$ and fermion fields. Taking the muon as an example, the interaction term and the term that gives mass to muon in the Lagrangian is shown as following:
\begin{equation}\label{Higgsmuonmass}
L_{\mu}=-\lambda_{\mu}(\bar{v}_{\mu},\bar{\mu}_{L})\phi\mu_{R}
            =-\frac{1}{\sqrt{2}}\lambda_{\mu}(v+H)\bar{\mu}_{L}\mu_{R}+ \cdots
\end{equation}
The mass of muon can be expressed as $M_{\mu}=\frac{\lambda_{\mu}v}{\sqrt{2}}$. Other fermions acquire masses in the same way. The coupling of Higgs and fermions can also be derived from Equation.~\ref{Higgsmuonmass}~\cite{DJOUADI20081}. 

The production modes of the Higgs boson at LHC are the gluon-gluon fusion(ggH), the vector boson fusion(VBF), the associated production with a vector boson(VH) and the production in association with a pair of top quarks($t\bar{t}H$). The Feynman diagrams for these production modes are shown in Figure.~\ref{fig:SM_H_production}. In LHC, proton proton collision, the real collisions are among quarks and gluons. ggH holds the biggest Higgs production cross section, while the VBF follows. The other two are relatively small compared with ggH and VBF. VBF and ggH are the signal production modes considered in the analyses that are presented in the later chapters. 

\begin{figure}[htbp] 
\centering
\includegraphics[width=0.8\textwidth]{chapter2/Higgs_production.jpg}
\caption{Production models in LHC}% \cite{Higgs_production_mode}}
\label{fig:SM_H_production}
\end{figure}

The cross section of the Higgs boson production modes in LHC have be calculated accurately. Besides the tree level diagrams that illustrated in the Figure.~\ref{fig:SM_H_production}, high order processes are needed. The inclusive cross sections of the Higgs boson production modes that are used in the 8 TeV and 13 TeV analysis are shown in Table.~\ref{cross_section_numbers}. These cross sections are calculated at NNLO at QCD and NLO electroweak accuracies for all of the production modes, besides ggH at 13 TeV, which is calculated at N3LO QCD and NLO electroweak accuracies. The accuracies here refers to the correction in high order terms, especially the QCD radiative corrections and electroweak correction~\cite{Dittmaier:1318996}. The high order correction in ggH and VBF production modes are discussed in the following. In the ggH production mode, the Higgs boson couples to gluons through quark loops and the dominant contributions are from heavy quarks. The corrections in this mode includes vertex corrections, real gluon radiations, initial state rescattering and etc. Dealing with these corrections, the technique like large-$m_{t}$ limits and full consideration of top and bottom quark mass dependence have been used in the NLO corrections~\cite{Djouadi:1991tka}. In the same manner, NNLO correction has been calculated and further improved by resumming the soft-gluon contributions~\cite{Catani:2003zt}. In 13 TeV, N3LO correction in the ggH inclusive production mode has been computed in the effective theory in which the top-quark is integrated out~\cite{deFlorian:2227475}. The next important correction in the ggH production mode is the electroweak correction with two loop diagrams and the mixed QCD-EW effects have been computed~\cite{Anastasiou:2008tj}. A most update evaluation of the EW correction can be found in~\cite{deFlorian:2227475}. The uncertainties in the ggH cross section computation has two primary origins, the missing terms in the cross section calculation and the limited knowledge of PDF. In the VBF production mode, the NNLO QCD correction is computed by the structure function approach, which view the process as a double deep-inelastic scattering process and the two vector boson are emitted and merged into Higgs boson~\cite{Bolzoni:2010xr}, while the EW correction is with the Monte Carlo programs~\cite{Dittmaier:1416519}. 
 
%Large M-t the QCD corrections which have been calculated for Higgs masses sufficiently below the top threshold, can be applied approximately up to Higgs mass values for which the bending of the Born cross section signals possibly important threshold effects.

\begin{table}[htp!]
\caption{SM Higgs(m=125 GeV) production cross sections at $\sqrt{s}=8$ TeV and 13 TeV. The ttH production is at NLO QCD and NLO electro-weak accuracies while other production models listed here are at NNLO QCD and NLO electro-weak accuracies~\cite{Yellow_report}}
\begin{center}
\begin{tabular}{|c|c|c|c|c|c|}\hline
           & ggH(pb)             &  VBF(pb)      &   WH(pb)      &   ZH(pb)          & ttH(pb)\\\hline
8 TeV  &  1.947E+01       & 1.601E+00    &  7.026E-01  &   4.208E-01    & 1.330E-01      \\\hline
13 TeV &   4.414E+01      &  3.783E+00   &  1.373E+00 &   8.839E-01    &  5.071E-01     \\\hline
\end{tabular}
\end{center}
\label{cross_section_numbers}
\end{table}

%\begin{center}
%\begin{tabular}{|c|c|c|c|c|c|}\hline
%           & ggH(pb)             &  VBF(pb) \\\hline
%8 TeV  &  1.947E+01        & 1.601E+00      \\\hline
%13 TeV &   4.414E+01      &  3.783E+00     \\\hline
%\end{tabular}
%\end{center}
%\label{cross_section_numbers}
%\end{table}








\section{Lepton flavour violation in beyond stand model theories}
Lepton flavour violating Higgs decays are forbidden in SM. The relevant terms of LFV in SM Lagrangian will not be compatible with renormalization under local gauge symmetry. But beyond Standard Model, Lepton flavour conservation does not necessarily hold. In the following, two methods that can introduce LFV are presented, the model independent effective field approach and the two Higgs doublet model. In both cases, the focus is how the LFV Higgs decay shows up in the new theories. 


\subsection{Lepton flavour violation Higgs decay in effective field theory}\label{effective_field}
The Effective Lagrangian approach is widely used to explore the new physics with higher dimensional operators in a model independent way. The effective Lagrangian in the theory can be written as up to dimension six operators: 
\begin{equation}
L_{eff}=L_{SM}+\sum\frac{a^{ij}_{n}}{\Lambda^{2}}O^{ij}_{n}
\end{equation}
where $a^{ij}_{n}$ is the coefficients, $i,j,(=1,2,3)$ are flavour indices, n holds the number of the operators and $\Lambda$ is the new physics scale. 

Various effective field operators can introduce processes that violate lepton flavour conservation~\cite{PhysRevD.62.116005}. As an example, the Yukawa-type operators that generate LFV Higgs decays will be discussed. The Yukawa-type operators hold the following terms:
\begin{equation}
O^{ij}_{L\phi}=(\Phi^{\dagger}\Phi)(\bar{L}_{L_{i}}l_{R_{j}}\Phi)
\end{equation}
While, inside the Lagrangian, the change to SM Lagrangian takes the form:
\begin{equation}
\Delta L_{Y}=-\frac{\lambda^{'}_{ij}}{\Lambda^{2}}(\Phi^{\dagger}\Phi)(\bar{L}^{i}_{L}l^{j}_{R}\Phi)+h.c...
\end{equation}
Taking in the scalar doublet expansion around the VEV as shown in Equation.~\ref{Higgs_vev_expansion}, the O(6) Yukawa-type operators have the following form:
\begin{equation}
\begin{aligned}
\Delta L_{Y}=&-\frac{\lambda^{'}_{ij}}{\Lambda^{2}}(\Phi^{\dagger}\Phi)(\bar{L}^{i}_{L}l^{j}_{R}\Phi)+h.c...\\
         =&-\frac{\lambda^{'}_{ij}}{2\sqrt{2}\Lambda^{2}}l_{L}^{i}l_{R}^{j}(v+H)^{3}+h.c...\\
         =&-\frac{\lambda^{'}_{ij}v^{3}}{2\sqrt{2}\Lambda^{2}}l_{L}^{i}l_{R}^{j}-\frac{\lambda^{'}_{ij}3v^{2}}{2\sqrt{2}\Lambda^{2}}l_{L}^{i}l_{R}^{j}+h.c...
\end{aligned}
\end{equation}
%The new Yukawa coupling terms have effects on fermion masses and SM Yukawa interactions. 
The SM Higgs and lepton coupling components in Equation.~\ref{Higgsmuonmass} is obtained through the diagonalization of mass matrices~\cite{Harnik:2012pb}. The total Lagrangian in the effective field theory is a combination of SM Lagrangian $L_{SM}$ and the effective field Lagrangian $\Delta L_{Y}$. Thus the combined fermion mass and Yukawa interaction terms are following:
\begin{equation}
\sqrt{2}m=V_{L}\Big[\lambda+\frac{v^{2}}{2\Lambda^{2}}\lambda^{'}\Big]V^{\dagger}_{R},~~\sqrt{2}Y=V_{L}\Big[\lambda+3\frac{v^{2}}{2\Lambda^{2}}\lambda^{'}\Big]V^{\dagger}_{R}
\end{equation}
The Yukawa couplings of Higgs and leptons mixed in the contributions from dimension six operators and have the following form, in which $\hat{\lambda}=V_{L}\lambda^{'}V_{R}$
\begin{equation}
Y_{ij}=\frac{m_{i}}{v}\delta_{ij}+\frac{v^{2}}{\sqrt{2}\Lambda^{2}}\hat{\lambda}_{ij}
\end{equation}
In the limit $\Lambda \to \infty$, the SM results can be recovered, but in general case, like in the electro-weak scale and a arbitrary non-diagonal matrix $\hat{\lambda}$, LFV Higgs decays can be introduced into the theory through effective fields.




\subsection{LFV in two Higgs models}

In the model with two Higgs doublets(2HDM) $\Phi_{1}$ and $\Phi_{2}$, similar to the SM, the Yukawa interaction can be written as: 
\begin{equation}
L=y_{1}\bar{L}\Phi_{1}E+y_{2}\bar{L}\Phi_{2}E+h.c,
\end{equation}
Here, $y_{1}$ and $y_{2}$ are Yukawa couplings. If there is not a parity symmetry distinguish the two Higgs doublets, there can be coupling of the two doublets to the leptons in the tree level. In general it is impossible to diagonalize $y_{1}$ and $y_{2}$ simultaneousl, thus the LFV Higgs decay can be presented in the renormalizable Lagrangian~\cite{deLima2015}.% A bit more description of 2HDM is in the following~\cite{BRANCO20121}.   


In 2HDM, there are in general four sub-type of models, the main difference comes from the coupling of Higgs doublets to the quarks and leptons as shown in Table.~\ref{2HDM_models}.
\begin{table}[htp!]
\caption{Four types of 2HDM models differs by the coupling to Higgs doublet fields}
\begin{center}
\begin{tabular}{|c|c|c|c|}
\hline
Model                                 &~  $u^{i}_{R}$~  &~  $d^{i}_{R}$ ~   &~   $e^{i}_{R}$ ~\\\hline
Type I                                 &  $\Phi_{2}$    &  $\Phi_{2}$     &  $\Phi_{2}$   \\\hline
Type II                                &  $\Phi_{2}$    &  $\Phi_{1}$     &  $\Phi_{1}$   \\\hline
Type III (lepton specific)     &  $\Phi_{2}$    &  $\Phi_{2}$     &  $\Phi_{1}$   \\\hline
Type IV (Flipped)                &  $\Phi_{2}$    &  $\Phi_{1}$     &  $\Phi_{2}$   \\\hline
\end{tabular}
\end{center}
\label{2HDM_models}
\end{table}
The type III 2HDM is more relevant to this dissertation which can have the tree level LFV Higgs decay. The Higgs doublets can have the general form:
\begin{equation}
\Phi_{j}=
\begin{pmatrix}
\phi^{+}_{j}    \\
(v_{j}+\phi_{j}+i\eta_{j}/\sqrt{2})
\end{pmatrix}
\end{equation}
Under the common assumption of CP conservation in the Higgs sector and not spontaneously broken, the quartic odd terms are eliminated in the potential, then the scalar potential can be expressed as
\begin{equation}
\begin{aligned}
V=&m^{2}_{11}\Phi^{\dagger}_{1}\Phi_{1}+m^{2}_{22}\Phi^{\dagger}_{2}\Phi_{2}-m^{2}_{12}(\Phi^{\dagger}_{1}\Phi_{2}+\Phi^{\dagger}_{2}\Phi_{1})+\frac{\lambda_{1}}{2}(\Phi^{\dagger}_{1}\Phi_{1})^{2}+\\
  &\frac{\lambda_{2}}{2}(\Phi^{\dagger}_{2}\Phi_{2})^{2}+\lambda_{3}\Phi^{\dagger}_{1}\Phi_{1}\Phi^{\dagger}_{2}\Phi_{2}+\lambda_{4}\Phi^{\dagger}_{1}\Phi_{2}\Phi^{\dagger}_{2}\Phi_{1}+\frac{\lambda_{5}}{2}[(\Phi^{\dagger}_{1}\Phi_{2})^{2}+(\Phi^{\dagger}_{2}\Phi_{1})^{2}]
  \end{aligned}
\end{equation}
The scalar doublets fields $\Phi_{1}$ and $\Phi_{2}$ are not physical observables but the mass eigenstates. So any combination of the scalar doublet fields, as long as it preserve CP and solid gauge symmetries in SM, produces the same physics results~\cite{BRANCO20121}. The following based is referred as Higgs basis for the Higgs doublets:
\begin{equation}\label{Higgs_base}
H_{1}=
\begin{pmatrix}
G^{+}    \\
\frac{1}{\sqrt(2)}(v+\phi^{0}_{1}+iG^{0})
\end{pmatrix}~,~
H_{2}=
\begin{pmatrix}
H^{+}    \\
\frac{1}{\sqrt(2)}(\phi^{0}_{2}+iA)
\end{pmatrix}
\end{equation}
The relationship between scalar field $\phi_{1}^{0}$ and $\phi_{2}^{0}$ and neutral Higgs mass eigenstates h and H is the following:
\begin{equation}\label{Higgs_mass_states}
\begin{aligned}
h=&sin(\alpha-\beta)\phi_{1}^{0}+cos(\alpha-\beta)\phi_{2}^{0}\\
H=&cos(\alpha-\beta)\phi_{1}^{0}-sin(\alpha-\beta)\phi_{2}^{0}
\end{aligned}
\end{equation}
The angle $\alpha-\beta$ is the mixing angle between these two groups of scalars. The interactions of Higgs and fermions are through Yukawa coupling. In the Higgs base, the Yukawa interaction terms in the Lagrangian of the 2HDM can be expressed as:
\begin{equation}
\begin{aligned}
-L_{Y}=&\sqrt{2}\big(\bar{q}_{L_{j}}\tilde{H}_{1}\frac{K^{\ast}_{ij}m^{U}_{i}}{v}u_{R_{i}}+\bar{q}_{L_{i}}H_{1}\frac{m^{D}_{i}}{v}d_{R_{i}}+\bar{l}_{L_{i}}H_{1}\frac{m^{E}_{i}}{v}e_{R_{i}}\big) \\
            &+\bar{q}_{L_{i}}\tilde{H}_{2}\rho^{U}_{ij}u_{R_{j}}+\bar{q}_{L_{i}}H_{2}\rho^{D}_{ij}d_{R_{j}}+\bar{l}_{L_{i}}H_{2}\rho^{E}_{ij}e_{R_{j}}+h.c..
\end{aligned}
\end{equation}
Inside the Lagrangian, the terms denote as following, $\tilde{H}_i=i\sigma_{2}H^{\ast_{i}}$,  $K_{ij}$ as the CKM matrix and $\rho^{U,D,E}$ are complex matrices in flavor space. Taking in Equation.~\ref{Higgs_base} and \ref{Higgs_mass_states}, the lepton Yukawa interaction terms are collected as:
\begin{equation}
\begin{aligned}
-L_{Y}=&\bar{e}_{i}\big(\frac{m^{E}_{i}}{v}\delta_{ij}s_{\beta-\alpha}+\frac{1}{\sqrt{2}}\rho^{E}_{ij}c_{\beta-\alpha}\big)e_{j}h\\
            &+\bar{e}_{i}\big(\frac{m^{E}_{i}}{v}\delta_{ij}c_{\beta-\alpha}-\frac{1}{\sqrt{2}}\rho^{E}_{ij}S_{\beta-\alpha}\big)e_{j}H+...
\end{aligned}
\end{equation}
Term $s_{\beta-\alpha}$ and $c_{\beta-\alpha}$ stand for $sin(\beta-\alpha)$ and $cos(\beta-\alpha)$ respectively. The coupling of leptons and Higgs in the 2HDM model typeIII are expressed as
\begin{equation}
\begin{aligned}
g_{hff^{'}}&=\frac{m_{f}}{v}s_{\beta-\alpha}\delta_{ff^{'}}+\frac{\rho_{ff^{'}}}{\sqrt{2}}c_{\beta-\alpha}\\
g_{Hff^{'}}&=\frac{m_{f}}{v}s_{\beta-\alpha}\delta_{ff^{'}}-\frac{\rho_{ff^{'}}}{\sqrt{2}}c_{\beta-\alpha}\\
\end{aligned}
\end{equation}
So this shows the possibility of LFV Higgs decay at tree level~\cite{PhysRevD.90.115004}.








